\section{Complexity comparison}
\label{sec:complexity}
The complexity of \Agglomerate (Algorithm~\ref{algo:fuse}) is mainly due to the computation of the
minimum norm base in line~\ref{ln:MNB:1}.  This computation is repeated at most $n$ times.  With
$\op{MNB}(n)$ denoting the complexity of the minimum norm base algorithm for a ground set of size $n$,
\Agglomerate runs in time $O(n\op{MNB}(n))$. Since the agglomerative info-clustering algorithm in
Algorithm~\ref{alg:aic} invokes function \Agglomerate $N-1\leq n-1$ times, it runs in time $O(n^2
\op{MNB}(n))$, which is equivalent to that of~\cite[Algorithm~3]{chan16cluster}, assuming that the
submodular function minimization therein is implemented by the minimum norm base algorithm, i.e.,
with $\op{SFM}=\op{MNB}$. In contrast, the divisive info-clustering algorithm
in~\cite[Algorithm~2]{chan16cluster} makes $N-1$ calls to a subroutine that calculates the
fundamental partition. However, computing the fundamental partition appears to take time $O(n^2
\op{MNB}(n))$, which leads to an overall complexity of $O(n^3 \op{MNB}(n))$ for the divisive
clustering.


